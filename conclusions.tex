\chapter{Conclusions and outlook
\label{ch:conclusions}}
In this work two new approximate coupled cluster methods were developed. Our 
approach is based on tensor decompositions, which are alternative 
representations of high dimensional tensors ubiquitous in post-Hartree-Fock 
theories. After an introduction into the theory of coupled cluster is done, we 
focus on three tensor formats employed in this work: resolution of 
identity (RI), canonical polyadic decomposition (CPD) and tensor 
hypercontraction (THC). We represent mentioned tensor formats in a uniform way 
with the help of tensor diagrams and discuss techniques to cast different 
index-based tensors into CPD and THC forms. 

While higher order tensors are approximated with tensor decompositions, the 
decisive quantity is the length of the expansion (the rank). The cost of 
tensor contractions with decomposed tensors depends on the form of the 
decomposition, the rank(s) $r$ and the basis size $N$. We 
demonstrate that intermediates in coupled cluster can be calculated at 
cost scaling as the 4-th power of $r$ and $N$. The discussion culminates by 
the derivation of a general tensor structured coupled cluster method using THC 
factorization as an example. In our approach the Alternating Least Squares 
procedure is combined with the coupled cluster update to produce a new 
iterative algorithm for the factors of the decomposed cluster amplitudes. This 
generic algorithm has quartic cost per iteration if the THC or the CPD 
factorization is used for the amplitude tensors.

After establishing a general framework for tensor structured coupled cluster, 
we present its first concrete realization, namely THC-RCCSD. We demonstrate 
that only small ranks of order $N^{1.3}$ in the THC decomposition of two-body 
interaction are sufficient for sub-mHartree accuracy in MP2 energy, and 
that the quality of the decomposition does not depend significantly on the 
choice of the basis. After combining THC factorized two-body interaction with 
THC factorized amplitudes, we demonstrate that only small ranks of order 
$N^{1.3}$ are needed for the sub-mHartree accuracy of the resulting THC-RCCSD 
method. Provided these findings, the overall scaling of THC-RCCSD is of order 
$O(N^{5})$ for sub-mHartree accuracy, whereas the original RCCSD method has 
$O(N^{6})$ computational cost. The method is tested on a large set of organic 
molecules.   

We then introduce another combination of tensor decompositions to approximate 
RCCSD. The two-body interaction is represented in a standard RI form, 
while cluster amplitudes are treated in CPD format. The resulting CPD-RCCSD 
procedure again has a quartic cost in the ranks and basis, but also is superior 
to the THC-based method due to the use of the fast RI factorization. We prove 
that the sub-mHartree accuracy is obtained with ranks of order $O(N^{1.3})$, 
and hence the overall cost of CPD-RCCSD is similar to THC-RCCSD. We found that 
the accuracy of CPD-RCCSD is around $5$ times lower than that of THC-based 
method with the same rank, but the number of parameters is much less and 
scales only linearly with rank and $N$. Overall, CPD-RCCSD presents a faster and 
simpler alternative to THC-RCCSD. 

Finally, we have demonstrated that the failure of coupled cluster in strong 
correlation regime can be mitigated by using low rank decompositions of 
cluster amplitudes. We argue that low-rank approximations provide a way to 
regularize ill-posed residual equations and draw analogies from methods 
motivated by similar ideas. The later provides a direction for future 
research.

As the framework we developed is quite general, many extensions are possible. 
Tensor structured coupled cluster can be directly applied to unrestricted or 
generalized coupled cluster theories. The mentioned methods work for strongly 
correlated molecular systems at the expense of breaking the spin symmetry of 
the wavefunction. CPD-RCCSD can be directly generalized to the higher orders of 
coupled cluster theory, for example CCSDT, CCSDTQ, CCSDTQ5, where much larger 
numerical savings may be obtained. Additionally, the techniques we describe 
can be used in other CC-like approaches, such as the perspective polynomial 
similarity transformation methods.\cite{degroote2016polynomial, 
gomez2017attenuated} Finally, the regularizing effect of 
low-rank amplitudes calls for the study of regularization methods in the 
coupled cluster approach.