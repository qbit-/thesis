\thispagestyle{empty}
\begin{abstract} 
A constant goal of quantum chemistry is devising accurate and computationally 
effective methods for molecular simulations. In this work an application of 
tensor decompositions in the context of highly accurate Coupled Cluster theory, 
which is often considered a "gold standard", is investigated. The scheme we 
develop is aimed to mitigate a steep growth of the computational cost with the 
system size, and hence to overcome the "curse of dimensionality" common 
in many potent methods in electronic structure. We show how to reduce the 
computational effort of the Restricted Coupled Cluster with Singles and Doubles 
(RCCSD) by two orders of magnitude by introducing alternative parameterizations 
of the method using "canonical polyad decomposition" (CPD) and "tensor 
hypercontraction" (THC) formats. After describing CPD and THC formats in 
detail, we demonstrate how to cast regular index based tensors into a decomposed 
form. The number of parameters and the accuracy of these representations depends 
on the expansion length (rank) of the approximation. We investigate the 
dependence of rank upon the size of the system and a target accuracy and show it 
to be low for typical tensors in electronic structure. We then provide a 
generic procedure to reformulate any Coupled Cluster method using tensor 
decompositions. 
Two specific approximate methods, THC-RCCSD and CPD-RCCSD, are derived. We 
demonstrate the accuracy of these new approaches by calculating energies of a 
large set of organic molecules, as well as by simulations of Hubbard models. 
Finally, it is shown how the restriction of the number of parameters in 
approximate Coupled Cluster can improve the accuracy in the challenging strong 
correlation regime. We conclude by discussing a connection of our findings to 
other new developments in the Coupled Cluster theory and propose possible 
extensions of our approach.
\end{abstract}


