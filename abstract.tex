\thispagestyle{empty}
\begin{abstract} 

A constant goal of quantum chemistry is devising accurate and computationally 
effective methods for molecular simulations. The accuracy of chemical 
predictions ultimately depends on the ability of a method to model electronic 
correlation for all ranges of parameters of the molecular systems. While 
effective tools exist, they often have unfavorable computational demands, are 
not systematically improvable or work well only in a particular region of 
molecular parameters, such as equilibrium geometry. In this thesis we present 
our work on two perspective approaches to capture electronic correlation with 
computational cost scaling only quartically in the system size.

First, we extend the application of the Projected Hartree-Fock (PHF)
method, which provides an effective description of electron correlation both 
near and far from the equilibrium. To make it practical for chemical 
simulations, energy derivatives and dipole moments need to be evaluated.
We derive analytical energy gradients for the single-reference PHF theory 
and use it to calculate equilibrium geometries and harmonic frequencies of 
sizable molecules.

A way to systematically improve the PHF method is to increase the number of 
reference states in the calculation. The resulting approach may reach exact 
solution, but is much more demanding numerically. Fortunately, numerically 
intensive parts of the multi-reference PHF can be effectively 
parallelized on modern computers. We implemented parallel multi-reference PHF 
and present simulations of very large Hubbard systems. 
 
Our second line of work is a family Coupled Cluster methods, often considered 
a "golden standard" of quantum chemistry. We show how to reduce the 
computational effort of the Restricted Coupled Cluster with Singles and 
Doubles by two orders of magnitude by introducing tensor decompositions. 
We demonstrate the accuracy of our low-cost approach by benchmark 
calculations on various organic molecules.

An interesting property of our new Tensor Structured Coupled Cluster is the 
ability to cure the failure of the original Coupled Cluster far from 
equilibrium geometries, or in strong correlation regime. We illustrate this by 
simulations of molecular dissociation and Hubbard systems. Lastly, we discuss 
the extension of our scheme to other methods in the Coupled Cluster family. 
 
\end{abstract}


