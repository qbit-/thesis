\chapter{Introduction
\label{ch:introduction}}
In recent 50 years electronic structure methods evolved into powerful tools 
used in many areas of research, including biochemistry, organic 
synthesis and catalysis, materials design and astrophysics.~\cite{} The 
electronic structure simulations help to understand many processes not easily 
accessible for experimental study, such as short-living transition complexes in 
chemical reactions, as well as to simulate exotic states of matter, such as 
superconductivity.~\cite{} An ever increasing demand for reliable 
simulations requires a constant increase of accuracy of existing tools, as 
well as the need to simulate larger and larger systems. 

Nowadays some successful approaches like density functional theory 
(DFT) or second order many-body perturbation theory (MP2) can be routinely used 
for the calculation of large size systems (200-1000 atoms). While being 
extremely efficient, these theories are often limited in their accuracy. 
Another problem of these efficient methods is their failure to 
describe many systems of interest, like transition states, open-shell molecules 
and states of matter caused by significant quantum effects, such as Mott 
insulators. The correlated behavior of electrons, which lays at the root of the 
description of challenging quantum systems, proved hard to be simulated in 
electronic structure. While a wide range of tools was developed over the 
years,~\cite{} they are still limited to small systems due to a very steep 
scaling of computational cost, and often suffer from numerical problems. Thus 
the search for effective methods capable to capture electron correlation at all 
ranges of molecular parameters remains an active topic of modern 
computational chemistry.

Traditional methods are based on configuration interaction (CI)-like expansion 
around the Hartree-Fock (HF) solution. Some of them, like the coupled cluster 
(CC) theory have evolved into highly efficient tools,\cite{benedict_intro} and 
their accuracy and robustness have been proven in many 
studies.\cite{benedict_intro} However, the steep scaling of computational cost 
limits the application of coupled cluster to small systems. The cost of the 
highly popular CCSD(T) method scales as $O(N^7)$, where $N$ is a measure of the 
systems size. This means that for a twice large system the effort to get 
solution is 128 times higher. The cause of the high cost of CC theories is a 
large number of optimization parameters they use, which scales as $O(N^6)$ for 
CCSD(T). This "curse of dimensionality" is a hallmark of methods based on 
orthogonal expansions around Hartree-Fock solutions.
Additionally, CC methods have difficulties at strong correlation when the HF 
solution becomes qualitatively incorrect. Alternative formulations of coupled 
cluster which used different reference functions~\cite{} have exponential 
scaling of computational cost and other numerical problems~\cite{} after decades 
of research.

However, there are many indications that traditional approaches are heavily 
overparameterized: the electron correlation in molecules decays as 
$\frac{1}{r^6}$, and hence the number of parameters should not grow 
exponentially, but rather linearly with the size of the system. One such 
indication is a family of recently developed Projected Hartree-Fock (PHF)
methods by Scuseria, Jim{\'e}nez-Hoyos and coworkers.~\cite{} In contrast to 
traditional approaches, PHF methods use a non-orthogonal expansion to 
parameterize the solution, while having a formal cost of the Hartree-Fock. While 
excelling at strong correlation, PHF methods, however, have a lower overall 
accuracy than coupled cluster approach. Merging the ideas of PHF and CC theories 
is an active topic of research.~\cite{}

This work pursues two related lines of research. First, we extend the 
application of the Projected Hartree-Fock for practical applications. The 
calculation of energy derivatives and dipole moments is often required in 
simulations, for example, to estimate equilibrium structure of molecules and 
their vibrational spectra. We derive analytical energy gradients for the 
single-reference PHF theory, which allows efficient calculation of the said 
quantities. Our development is supplied with exemplary calculations of 
some sizable molecules.

The second line of work is devoted to traditional coupled cluster 
methods. Coupled cluster parameterize the solution in terms of cluster 
amplitude tensors, the multidimensional arrays of numbers responsible for the 
implausible cost of CC methods. We show how to reduce the computational effort 
of the restricted coupled cluster with singles and doubles (RCCSD) by two 
orders of magnitude by introducing new representations for the cluster 
amplitudes. By using tensor decompositions high order amplitude tensors are 
approximated by sets of matrices, which is a key in reducing the cost of 
coupled cluster. We demonstrate the accuracy of our low-cost approach by 
benchmark calculations on various organic molecules. Lastly, we discuss other 
interesting interesting properties our approach may have.

In the next chapter a short review and theoretical background on the 
electronic structure problem is given. In 
chapter~\ref{ch:projected_hartree_fock} we provide a more in-depth 
overview of the Projected Hartree-Fock method, followed by the derivation of 
its energy gradient expression. We conclude with an example 
calculation of geometries and harmonic spectra of  benzine 
biradicals. In chapter~\ref{ch:tcc} we discuss the cause of high 
computational cost of CC methods, followed by the idea of tensor 
decompositions. Chapter~\ref{ch:tcc} is concluded with a derivation of a 
general scheme for applying tensor decompositions in context of coupled cluster 
methods. Chapter~\ref{ch:app_tcc} presents some applications of the proposed 
methods. We estimate the accuracy of approximating different ingredients of 
coupled cluster equations and test approximate CC methods on a large set of 
molecules. Chapter~\ref{ch:app_tcc} concludes with a discussion of an 
additional interesting properties our methods possess. An overall conclusion 
and outlook is given in chapter~\ref{ch:conclusions}.


\chapter{Preliminaries
\label{ch:preliminaries}}

\section{The electronic structure problem}
\label{sec:motivation}
The goal of quantum chemistry is to simulate and predict the behavior 
of chemical systems. At the root of chemical phenomena is a collective motion 
of electrons in the field of the nuclei. Together, electrons form a quantum 
many-body system. If one would be able to efficiently simulate the energy and 
the state of this system numerically, many questions of chemistry could be 
answered.

Since the advent of quantum mechanics all ingredients of many-body simulations 
are set. One needs to find the eigenvectors and eigenvalues of the 
Hamiltonian describing the system:
%
\begin{equation}
 H | \Psi \rangle = E | \Psi \rangle
 \label{eq:schroedinger}
\end{equation}
%
where $|\Psi \rangle = \Psi(\vec{x_{1}}, \vec{x_{2}}, \vec{x_{3}}\ldots 
\vec{x_{M}})$ is a wavefunction describing the state of an $M$-electron system 
and $E$ is the energy of the state. The variables $\vec{x}$ describe the 
degrees of freedom of the electrons, like coordinates and spin.
The Hamiltonian is usually written in a short form, due to the 
equivalence of quantum particles:
%
\begin{equation}
 \hat{H}(\vec{x^{\star}_{1}}\ldots 
\vec{x^{\star}_{M}}, \vec{x_{1}}\ldots 
\vec{x_{M}}) = h^{0} + \hat{h}(\vec{x^{\star}}, \vec{x}) + 
\frac{1}{2} \hat{V}(\vec{x^{\star}}, \vec{y^{\star}}, \vec{x}, 
\vec{y}) \qquad \forall \vec{x}, \vec{y} \in \{\vec{x}_{1}, \ldots, \vec{x}_{M} 
\}
\end{equation}
%
Here $h^{0}$ is a constant, $\hat{h}$ is a one-body part, e.g. it acts on each 
electron individually, and $\hat{V}$ is a two body part, acting on every pair 
of electrons. 

To solve~\ref{eq:schroedinger} in practice a discrete basis with $N$ 
functions is introduced for each of the variables $\vec{x}$, the procedure 
mathematicians would call Galerkin discretization. We denote this new 
$N$-valued discreet spaces by $\mathcal{V}$. As the number of basis functions 
increases, the solution of the discretized problem will approach the true 
solution of~\ref{eq:schroedinger}. After standard approximations, like a 
Born-Oppenheimer approximation,~\cite{} the molecular Hamiltonian becomes:
%
\begin{equation}
 \hat{H} = h^{nr} + h_{\mu \nu} a^{\dagger}_{\mu} a_{\nu} + \frac{1}{2} V_{\mu 
\nu \lambda \sigma} a^{\dagger}_{\mu} a^{\dagger}_{\nu} a_{\sigma} a_{\lambda}
\label{eq:hamiltonian}
\end{equation}
%
We imply summation over repeated indices in the expression above and the rest 
of the text unless otherwise stated. In Eqn.~\ref{eq:hamiltonian} the constant 
$h^{nr}$ is a nuclear repulsion energy, the one-body part $h$ describes a 
kinetic energy of an electron and its attraction to the nuclei:
%
\begin{equation}
 h_{\mu \nu} = \int d\vec{x_{1}} ~ \mu^{\star}(\vec{x_{1}}) \left( - 
\frac{1}{2} \nabla^2 - \sum_{a} \frac{Z_{a}}{|\vec{x_{1}} - 
\vec{r_{a}}|} \right) \nu(\vec{x_{1}})
\end{equation}
% 
The functions $\mu(\vec{x}), \nu(\vec{x})$ are basis functions used for the 
discretization of the problem and $r_{a}$ and $Z_{a}$ are the 
position and the charge of the nuclei. The two-body, or electron repulsion 
part, is
%
\begin{equation}
 V_{\mu \nu \lambda \sigma} = \int \int d\vec{x_{1}} d\vec{x_{2}}~ 
\mu^{\star}(\vec{x_{1}}) \nu^{\star}(\vec{x_{2}}) 
\frac{1}{|\vec{x_{1}} - \vec{x_{2}}|} 
\sigma(\vec{x_{2}}) \lambda(\vec{x_{1}}) 
\end{equation}
% 
We denoted by $a^{\dagger}_{\mu}$ and $a_{\mu}$ creation and 
annihilation operators. For example, $a^{\dagger}_{\mu}$ creates a single 
particle state $a^{\dagger}_{\mu} | - \rangle = | \mu \rangle$, where $| - 
\rangle$ is a physical vacuum. Equivalently, $a^{\dagger}_{\mu}$ and $a_{\mu}$ 
can be thought as simply a $\mu$-th unit vector (or its adjoint) in the 
discrete space $\mathcal{V}$, and $| - \rangle$  as a zero vector.

Despite a simple structure of the Hamiltonian, even a discretized
Schr{\"o}edinger equation is hard to solve. The solution is contained in an 
antisymmetric product (due to Pauli exclusion 
principle) of single particle vector spaces. The resulting space, called 
Hilbert space of the problem, is: 
%
\begin{equation}
 \mathcal{H} = \underbrace{\mathcal{V} \times \mathcal{V} \times \mathcal{V} 
\times \ldots}_{M ~\mathrm{times}}
\end{equation}
%
The Hamiltonian is thus a matrix in $\mathcal{H} \times \mathcal{H}$. The total 
size of the eigenvector of the Hamiltonian $|\Psi \rangle$ is 
of order of $N^{M}$ (it is actually $N \choose M$ because $\mathcal{H}$ is an 
antisymmetric vector space). Even for relatively small $M$ and $N$ a direct 
calculation of $|\Psi \rangle$ becomes prohibitive. The main problem of 
theoretical chemistry is thus, how to build feasible and accurate approximations 
to the true eigenvectors of $H$.

A commonly used approach is to use the variational principle~\cite{} to 
calculate the energy expectation value $\tilde{E}$ of a test wavefunction 
$\tilde{\Psi}$:
%
\begin{equation}
 \tilde{E} = \frac{\langle \tilde{\Psi} | H | \tilde{\Psi} \rangle}{\langle 
\tilde{\Psi} |\tilde{\Psi} \rangle} 
\end{equation}
%
It can be proved that this expectation value for a normalized  
function is an upper bound of the exact energy:
%
\begin{equation}
 E \geq E_{\mathrm{exact}}
\end{equation}
Although not all approaches to solve \ref{eq:schroedinger} are variational, 
variational nature is a valuable property of any particular method. 

\section{Solving Schr{\"o}edinger equation}
\label{sec:solving}
\subsection{Exact solution}
As we mentioned before, the direct solution of Sch{\"o}edinger equation is hard 
to obtain. One may still, however, embark on this task for very small systems. 
Let us introduce a basis of all states with $M$ electrons in the Hilbert 
space $\mathcal{H}$. This basis will contain all possible products of $M$ 
creation operators. Due to the antisymmetry of the 
wavefunction for electrons:
%
\begin{equation}
  \Psi(\vec{x_{1}}, \vec{x_{2}}) = - \Psi(\vec{x_{2}}, \vec{x_{1}}) 
\end{equation}
%
those products will contain only the combinations of $a^{\dagger}_{\mu}$ 
without repetitions over the index $\mu$ (repeated index will necessary turn 
product to zero):
%
\begin{equation}
 | \phi \rangle = \prod_{k=1}^{M} a^{\dagger}_{\mu_{k}} |- \rangle  
 \label{eq:slater_determinant}
\end{equation}
%
In total there will be $N \choose M$ products of the
form~\ref{eq:slater_determinant}, usually called Slater determinants or 
configurations. The Hamiltonian matrix can be written in this basis and 
diagonalized:
%
\begin{equation}
\begin{aligned}
 H_{ij} &= \langle \phi_{i} | \hat{H} | \phi_{j} \rangle \\
 | \Psi_{i} \rangle &= U_{ij} |\phi_{j} \rangle
\end{aligned}
\end{equation}
%
The eigenvectors $| \Psi \rangle$ obtained in this way and their respective 
eigenvalues are exact solutions to the Schr{\"o}edinger equation in a given 
basis. When available, they are often used as reference values in 
evaluation of the quality of quantum chemistry algorithms. Direct 
diagonalization, however, is possible only for the smallest systems. The 
simulation of a 12-electron system, like two carbon atoms, in a mediocre basis 
set of 10 basis functions per electron will involve a diagonalization of a 
matrix of size $1.054 \cdot 10^{16}$ which is already at the border of today's 
computational capabilities.

One may notice, however, that the molecular Hamiltonian is almost diagonal in 
the $M$-particle Hilbert space, as only one- and two-body parts are non-zero. 
This provides grounds for an important approximate method we are going to 
consider next. 

\subsection{Hartree-Fock solution}
The Hartree-Fock (HF) method is a starting point of most many-body approaches.
The idea of Hartree-Fock is to find a single best determinant to 
approximate the wavefunction:
%
\begin{equation}
 |\Psi_{SD} \rangle = \prod_{k=1}^{M} c_k^{\dagger} | -\rangle, \qquad 
c_{k}^{\dagger} = C_{\mu k} \cdot a^{\dagger}_{\mu}
\label{eq:single_determinant}
\end{equation}
% 
the parameters of this \emph{ansatz} are contained in the basis transformation 
matrix $C$. The energy 
of the state~\ref{eq:single_determinant} is:
%
\begin{equation}
 E = \frac{\langle \Psi_{SD} | H | \Psi_{SD} \rangle}{\langle \Psi_{SD} | 
\Psi_{SD} \rangle}
 \label{eq:hf_energy}
\end{equation}
%
The solution is found by minimizing the energy in~\ref{eq:hf_energy}, 
while constraining matrix $C$ to be unitary. The later is equivalent to 
minimizing a lagrangian:
%
\begin{equation}
\begin{aligned}
L(C, \epsilon) = h_{\mu \nu} \rho_{\nu \mu} &+ \frac{1}{2} 
V_{\mu \lambda \nu \sigma} \rho_{\nu \mu } \rho_{\sigma \lambda}  + 
\epsilon_{ij}(C^{\ast}_{\mu i} S_{\mu \nu} C_{\nu j} - 1_{N \times N})\\
\rho_{\nu \mu} &= C^{\ast}_{\mu i} C_{\nu i} 
\label{eq:hf_lagrangian}
\end{aligned}
\end{equation}
%
Here $C$ is a basis transformation one is solving for, $\epsilon$ is a 
set of lagrange multipliers used to enforce the unitary constraint, $1_{N 
\times N}$ is an identity matrix and $S_{\mu \nu} = \langle a^{\dagger}_{\mu} 
a_{\nu} \rangle $ is an overlap matrix of the possibly non-orthogonal original 
basis vectors $a$. Let us note that Hartree-Fock method 
is variational, and hence provides an upper bound for
the energy.\cite{levine2000quantum} 

There are multiple ways of solving Hartree-Fock equations, which we will not 
discuss here, but rather present a final result of the most well-known method 
of Roothaan.~\cite{roothan_carlos} By taking a derivative 
of~\ref{eq:hf_lagrangian} with respect to $C$, one ends up with a matrix 
equation:
%
\begin{equation}
\begin{aligned}
 FC^{\ast} &= \epsilon C^{\ast} S\\
 F_{\mu \nu} &= h_{\mu \nu} + V_{\mu \lambda \nu \sigma} \rho_{\sigma \lambda} 
 \label{eq:hf_roothaan}
\end{aligned}
\end{equation}
%
The equation~\ref{eq:hf_roothaan}, which looks like an eigenvalue problem for 
the Fock matrix $F$, is non-linear in $C$. It is solved by iterations starting 
from a guess for the transformation $C$. 

The Roothaans equation amounts to an important interpretation of the 
Hartree-Fock method. Note that if there would be no interaction term in the 
Hamiltonian, the Hartree-Fock would correspond to the exact 
diagonalization, and a Slater determinant, which is an antisymmetric direct 
product of single particle wavefunctions, would be an exact 
eigenstate. The Fock matrix in Eqn.~\ref{eq:hf_roothaan}, thus is a diagonal
approximation of the Hamiltonian, where the actual two-particle interaction 
$\hat{V}$ is replaced by a one-body term $\hat{V}_{eff} = \hat{V} \hat{\rho}$. 
The HF solution describes the electrons as if they would not immediately repel 
each other through Coulombic interaction, but rather move 
independently in an average potential $\hat{V}_{eff}$ of all other electrons. In 
reality, however, the electrons avoid each other dynamically. This brings us to 
the concept of electronic correlation: the true solution would need to account 
for the correlated motion of electrons. We would discuss this in more detail in 
the section.~\ref{sec:electronic_correlation} 

Overall, the Hartree-Fock method is a cornerstone of quantum chemistry due to 
its simplicity and theoretical basis. The computational cost of Hartree-Fock is 
$O(N^4)$ floating point operations in its simplest variants (and much less 
in optimized formulations\cite{}), which makes it one of the most 
affordable methods in electronic structure theory. Not being widely used on its 
own in practical simulations, it remains a usual starting point of most 
many-body approaches.


\subsection{Electronic correlation
\label{sec:electronic_correlation}}

Put simply, the electronic correlation is a difference between the exact and 
Hartree-Fock solution, and correlation energy is a difference between HF and 
exact energies. We will classify correlation based on two 
sources of errors introduced by the Hartree-Fock method, however, this division 
is only relative:

\begin{itemize}
\item Dynamic or weak correlation arises from small deviations of 
the Hartree-Fock potential and a faithful two-body interaction. It can be 
interpreted as an effect of short-range instantaneous repulsion between 
electrons.
\item Static or strong correlation emerges when the Hartree-Fock potential 
significantly deviates from the real two-body interaction, e.g. when the 
Hamiltonian is not a diagonally dominant matrix. Static correlation 
manifests as a near or exact degeneracy of different Slater determinants, and 
the single determinant picture becomes incorrect.
\end{itemize}

Let us illustrate the regions where different types of error of the HF method 
dominate in the process of nitrogen dissociation. 
%
\begin{figure}[tb]
%\includegraphics[width=\columnwidth]{figures/intro/n2_hf_exact}
\caption{Dissociation of nitrogen molecule using cc-pVDZ basis set. RHF refers 
to restricted Hartree-Fock, exact is taken from the exact diagonalization data.
\label{fig:n2_hf_exact}}
\end{figure}
%
Near the equilibrium the difference of Hartree-Fock and and exact energies is 
mostly related to short-range dynamical electronic repulsion. When the $N_2$ 
bond is stretched, the bonding, $\sigma^{\ast}_{2p}$ and antibonding $\pi_{2p}$ 
orbitals become degenerate, leading to the major overestimation of 
dissociation energy by the HF method. Let us now provide some examples of 
different types of correlation and its relation to physical phenomena.

Most correlation in "normal" systems is dynamical. Typical examples 
are usual organic molecules at equilibrium, most metals, semiconductors 
etc. Despite HF accounts for around $99\%$ of electronic energy, proper 
description of dynamical correlation is crucial in quantitative calculations.
For example, even for atoms, HF excitation energies may be wrong by 
more than as $40\%$~\cite{irek_9}, which leads to incorrect prediction of 
ionization potentials. Though difficult to generalize, lack of 
correlation leads to too short bonds at equilibrium, incorrect bond angles, 
errors in the dipole moments and too high harmonic vibrational 
frequencies.~\cite{irek_11, irek_12} HF usually predicts too high reaction 
barriers, which can be understood by the problems of HF to describe stretched 
bonds. Finally, Hartree-Fock approach is unable to predict pure 
dispersion interaction~\cite{irek_13}, correlation bound anions~\cite{irek_14} 
and similar systems.  

Statically correlated systems are where the shortcomings of Hartree-Fock theory 
are most pronounced, as independent particle picture becomes inappropriate. 
Static correlation is very common in extended systems and materials 
where $d$- and $f$-electron shells of atoms interact. Among many phenomena 
caused by strong correlation are large resistivity changes, huge volume changes 
across phase transitions, heavy fermion behavior, large 
magnetoresistence and high temperature superconductivity.~\cite{irek_15, 
irek_16}. Strongly correlated systems are of great importance and constant 
theoretical interest. Accounting for strong correlation is challenging for 
modern electronic structure methods.
 
The community of quantum chemists worked for decades to build methods to 
for properly accounting for electronic correlation. Up to now all existing 
approaches are flawed in one way or another. Nevertheless, let us list some 
important features an ideal approximate solution should have, which we will be 
using for comparison of different theories through the discussion:

\begin{itemize}
 \item Ideally, a theorist should seek an method which works well in the 
whole range of the parameters of the problem, e.g. which would provide accurate 
solutions both in strong and weak correlation regimes, independently of the 
size of the system. 

\item The method should be computationally tractable, with polynomial, and 
preferably low, dependence of cost on the system size.

\item A good many-body theory should be systematically improvable to provide a 
trade-off between accuracy and computational cost.

\item The method should be size consistent, which means that the energy of two 
non-interacting fragments $A$ and $B$ must be the sum of energies of those 
fragments calculated independently
\begin{equation}
 E(A \overset{d \longrightarrow \infty}{--} B) = E(A) + E(B)
\end{equation}

\item The predicted energy should scale linearly with the size of the system, 
e.g. the energy of $k$ non-interacting particles, like Helium atoms far apart 
from each other, should equal to the sum of energies of individual particles.
\begin{equation}
 E(k \cdot \mathrm{He}) = k \cdot E(\mathrm{He})
\end{equation}
This property is called size extensivity and is important in thermodynamics, 
where extensive quantities scale with system size
\end{itemize}

Let us now will provide more specific details on the coupled cluster 
and projected Hartree-Fock methods in the next chapters. 

