\chapter{Introduction
\label{ch:introduction}}
In the last 50 years electronic structure methods evolved into powerful tools 
used in many areas of research, including biochemistry, organic 
synthesis and catalysis, materials design and 
astrophysics.~\cite{young2004computational} 
Electronic structure simulations help to understand many processes not easily 
accessible for experimental study, such as short-living transition complexes in 
chemical reactions, and to study exotic states of matter, such as 
superconductivity, topological insulators 
etc.~\cite{imada1998metal,qi2011topological} An ever increasing demand for 
reliable simulations requires a constant increase in both the accuracy of 
existing tools and their performance to simulate larger and 
larger systems. 

Nowadays some successful approaches like density functional theory 
(DFT) or second order many-body perturbation theory (MP2) can be routinely used 
for the calculation of large systems (200-1000 atoms). While 
extremely efficient, these theories are often limited in their accuracy. 
Another problem of these low-cost methods is their failure to 
describe many systems of interest, like transition states, open-shell molecules 
and materials with significant quantum effects, such as Mott insulators. The 
correlated behavior of electrons, which lays at the root of the 
description of challenging quantum systems, has proven to be difficult to 
simulated in electronic structure. While a wide range of accurate tools has been
developed over the years, they are still limited to small systems due to a very 
steep scaling of computational cost, and often suffer from numerical problems. 
Thus the search for effective methods remains an active topic of modern 
computational chemistry.

Traditional approaches are based on configuration interaction (CI)-like 
expansion around the Hartree-Fock (HF) solution. Some of them, like coupled 
cluster (CC) theory, have evolved into highly efficient 
algorithms,\cite{crawford2000introduction,tew2007electron} and their accuracy 
and robustness have been proven in many 
studies.\cite{bartlett2007coupled, larsen2000full, 
feller2001extended, tajti2004heat, heckert2006basis, rauhut2009accurate} 
However, the steep scaling of computational cost limits the application of 
coupled cluster to small systems. 
The cost of the highly popular CCSD(T) method scales as $O(N^7)$, where $N$ is a 
measure of the system's size. This means that for a system twice as large 
the time to get solution is 128 times higher. The cause of the high cost of CC 
theories is the large number of optimization parameters they use, which grows 
as $O(N^4)$ for CCSD(T). This "curse of dimensionality" is a hallmark of 
theories based on orthogonal expansions around Hartree-Fock solutions.
Additionally, CC methods have difficulties at strong correlation when the HF 
solution becomes qualitatively incorrect. Alternative formulations of coupled 
cluster which used better reference 
functions~\cite{jeziorski1981coupled,mukherjee1989use} have exponential 
scaling of computational cost and other numerical 
problems~\cite{lyakh2011multireference} after decades 
of research.

The root of the problem of high computational cost in electronic structure is 
the need to manipulate tensors, which are high dimensional arrays of numbers. 
Quite often, however, the entries in physically motivated tensors are not 
independent, and a more compact representation with lower-dimensional tensors is 
possible. One of the first applications of this idea is "resolution of 
identity" (RI)~\cite{vahtras1993integral,weigend1998ri,hattig2000cc2, 
ahlrichs2004efficient} or "density fitting" 
(DF)~\cite{werner2003fast,schutz2003linear,manby2003density} which emerged in 
context of DFT more than 40 years ago, and which was widely used in explicitly 
correlated methods 
later.~\cite{kutzelnigg1985r,rychlewski2013explicitly,valeev2000evaluation} 
Closely related to RI/DF are methods based on 
Choleski decomposition (CD).~\cite{weigend2009approximated} RI/DF decompositions 
approximate the 
four index Coulomb interaction potential with lower dimensional objects, making 
it possible to perform index summations occurring in electronic structure 
calculations with lower complexity. In the context of DFT this leads to an 
order of magnitude decrease in computational cost. 

Another type of tensor decomposition was introduced by 
Alml{\"o}~\cite{almlof1991elimination} in the context of MP2. With the help of 
a Laplace transformation, the energy denominator can be approximated by a 
factorizable expression, which also leads to significant cost savings. The 
Laplace-MP2 approach was extended to the calculation of periodic systems by 
Izmaylov and Scuseria.~\cite{izmaylov2008resolution}

Recently, in a series of works more general tensor decompositions were proposed 
in accurate post Hartree-Fock calculations. Hino~\emph{et al.} developed a 
lower cost CCSD(T) method by applying Singular Value Decomposition to ${}^3 T$ 
amplitudes.~\cite{hino2004singular} Benedikt \emph{et al.}~\cite{benedict_mp2, 
benedict_ccd} utilized 
canonical polyadic decomposition (CPD) in the context of the MP2 and CC 
doubles approaches, and discussed the application of matrix product 
states in 
post-HF calculations.~\cite{benedikt_mps} Hohenstein, Parrish, Martinez and 
coworkers developed a tensor hypercontraction (THC) format by applying CP 
decomposition to three-center integrals produced by the RI/DF procedure. Their 
DF-THC factorization allows MP2 calculations in quartic 
cost.~\cite{kokkila2015tensor} Later THC was extended by these and other 
authors to approximate 
CC theories~\cite{parrish2014communication, hohenstein2013tensor}, as well as 
to interaction potentials in nuclear physics.~\cite{parrish2013exact} While 
demonstrating the great value of tensor 
factorizations in post Hartree-Fock methods, none of these 
works developed the idea in context of coupled cluster to its full potential.

Our interest in this work is the application of tensor decompositions in
traditional coupled cluster with singles and doubles (CCSD). Efficient 
implementations of CCSD are important as it is the base of the famous "gold 
standard" CCSD(T) approach. Additionally, our interest in improving CCSD is 
motivated by a series of CC-like methods developed in the Scuseria 
group,~\cite{degroote2016polynomial, gomez2017attenuated, 
qiu2017projected2, hermes2017combining} which have potential to describe 
strongly correlated systems yet use Hartree-Fock as a reference function. Ideas 
developed for CCSD could be transferred to these novel techniques. In the 
present work we demonstrate how to apply tensor decompositions to cluster 
amplitudes in a uniform framework. We use two factorizations considered before, 
namely CP and THC, to decrease the cost of restricted CCSD (RCCSD) by two orders 
of magnitude. The scheme, however, is quite general and is not limited to the 
mentioned factorizations.  

In the next chapter a short review of the 
electronic structure problem is given. We introduce the concept of electronic 
correlation and discuss the ideas behind the coupled cluster method. In 
chapter~\ref{ch:tcc} we discuss the cause of high computational cost of CC 
methods, followed by an overview of tensor decompositions. 
Chapter~\ref{ch:tcc} is concluded with a derivation of a general scheme for 
applying tensor decompositions to cluster amplitudes. Chapter~\ref{ch:app_tcc} 
presents some applications of THC- and CPD-based approximate CC methods.  We 
estimate the accuracy of approximating different ingredients of coupled cluster 
equations and test THC- and CPD-RCCSD methods on a large set of 
molecules. Chapter~\ref{ch:app_tcc} concludes with a discussion of
additional interesting properties our approach may have. An overall conclusion 
and outlook is given in chapter~\ref{ch:conclusions}.
