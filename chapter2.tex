\chapter{Tensor structured Coupled Cluster}
\label{ch:tcc} 
The discussion presented in this section based on our published work, see 
Ref.~\cite{schutski2017tensor}, and also on new results intended for another 
publication.

\section{Motivation}
\label{sec:Introduction} 
Having seen that both strong and weak electronic correlation can be captured 
by methods based on the Projected Hartree-Fock, let us now turn to another 
powerful family of theories to solve many-body problems, namely, Coupled 
Cluster (CC). Since their introduction in the nuclear 
physics,\cite{coester1958bound, coester1960short} Coupled Cluster methods 
quickly became a "gold standard" of quantum chemistry due to their exceptional 
ability to capture weak electronic correlation, while having polynomial 
computational cost in basis size. Another attractive properties of most CC 
methods are size consistency and size extensivity.\cite{pople1978electron, 
bartlett1978many, crawford2000introduction, bartlett2007coupled} 

{\color{red} Insert 2 paragraphs here where accuracy of Coupled Cluster is 
discussed. Take it from Irek.}

Coupled Cluster methods parameterize the solution in terms of a reference 
wavefunction $| 0 \rangle$ and a set of excitation operators:
\begin{equation}
 | \phi \rangle  = \exp({}^1\hat{T} + {}^2\hat{T} + {}^3\hat{T} + \ldots) | 0 
\rangle
\end{equation}
where ${}^1\hat{T}, {}^2\hat{T}, \ldots$ are appropriate single, double, and 
higher order excitations from the reference state $| 0 \rangle$ (usually a HF 
solution). The method with n-body excitations is exact for n-electron systems. 
In practice, however, the excitation operator is truncated at doubles due to 
high computational cost, leading to Coupled Cluster with Singles and Doubles 
methods (CCSD).\cite{purvis1982full} Higher order excitations can be included 
approximately with the help of perturbation theory, such as in the highly 
accurate and widely used Coupled Cluster with Singles, Doubles and 
perturbative Triples (CCSD(T)) approach.\cite{bartlett1990non}

CC methods are also distinguished by the form of excitation operators, 
leading to Restricted (RCCSD),\cite{scuseria_ccsd} Unrestricted (UCCSD) and 
General (GCCSD) Coupled Cluster, in the same way as Hartree-Fock methods can be 
classified. A further analogy to HF method is that the solution of conventional 
Restricted Coupled Cluster method has proper $S^{2}$ and $S^{z}$ symmetry, but a 
wrong energy in case of strong correlation.

\begin{figure}[!ht]
\centering
 \rule{0.8\textwidth}{2cm}
 \caption{Restricted Coupled Cluster with Singles and Doubles in different 
correlation regimes. System: $N_{2}$, cc-pVDZ basis set. Exact is FCI solution.}
\end{figure}

This failure of Restricted CCSD in strong correlation regime is one (and 
probably most serious) downside of traditional Coupled Cluster theory. Another 
drawback is a steep growth of computational cost,\cite{noga1987full, 
scuseria1989coupled} which scales as $O(N^6)$ for CCSD, $O(N^7)$ for CCSD(T), 
and $O(N^8)$ for CCSDT, where $N$ is a size of basis. In the following we set 
to tackle both of these problems of conventional Coupled Cluster theory with 
tensor decompositions.

\section{Restricted Coupled Cluster with Singles and Doubles}
Let us start by describing the traditional Restricted Coupled Cluster 
\emph{ansatz}.\cite{bartlett2007coupled} The wavefunction is 
parameterized by an excitation operator $T$ and a single reference 
determinant $| 0 \rangle$:

\begin{equation}
 | \phi \rangle  = \exp(\hat{T}) | 0 \rangle = \exp({}^1\hat{T} + {}^2\hat{T} + 
{}^3\hat{T} + \ldots) | 0 \rangle
\end{equation}

In restricted version of Coupled Cluster theory excitation operators are spin 
adapted combinations of creation and annihilation operators of the form:

\begin{equation}
\begin{split}
 E_{i}^{a} & = \frac{1}{2} \cdot (a_{a, \uparrow}^{\dagger} a_{i, \uparrow} + 
a_{a, \downarrow}^{\dagger} a_{i, \downarrow}) \\
 {}^{1}\hat{T} & = {}^{1}T_{i}^{a} E_{i}^{a} \\
 {}^{2}\hat{T} & = \frac{1}{2} ~~ {}^{2}T_{ij}^{ab} E_{i}^{a} E_{j}^{b} \\
 \ldots
\end{split}
\end{equation}

Here indices $i, j$ represent particles, $a, b$ represent holes and summation 
is implied over repeated indices. The quantities denoted by $T$ are 
excitation amplitude tensors, e. g. two, four or higher dimensional arrays of 
numbers.

The excitation operator is truncated to a specific level, the most widely used 
choice being doubles, e. g. ${}^{2}\hat{T}$. With a wavefunction in the 
chosen form the Schr{\"o}edinger equation is

\begin{equation}
 H \exp(\hat{T}) |0 \rangle = E \exp(\hat{T}) |0 \rangle
\end{equation}

This equation is usually solved projectively by multiplying both sides of the 
expression by $\exp(-\hat{T})$. 
\begin{equation}
 \exp(-\hat{T}) H \exp(\hat{T)} | 0 \rangle = \bar{H} | 0 \rangle = E | 0 
\rangle
\label{eq:cc_lhs}
\end{equation}

From equation \ref{eq:cc_lhs} the energy can be extracted as:

\begin{equation}
 E = \langle 0 | \bar{H} | 0 \rangle
\end{equation}

To obtain excitation amplitudes, the similarity transformed Hamiltonian 
$\bar{H}$ is then projected onto the set of excited determinants on the left.
Denoting these determinants as $\langle {}^{1}Z |$, $\langle {}^{2}Z|$
for single, double etc. excitations, one comes to a set of equations

\begin{equation}
\begin{cases}
 \langle {}^{1} Z_{i}^{a} | \bar{H} | 0 \rangle = {}^{1}R_{i}^{a} = 0 \\
 \langle {}^{2} Z_{ij}^{ab} | \bar{H} | 0 \rangle = {}^{2}R_{ij}^{ab} = 0 \\
 \ldots
\end{cases}
\label{eq:cc_residuals}
\end{equation}

where for every excited determinant the corresponding value of the residuals 
$R$ is zero. Residual equations are polynomial in the excitation amplitude 
tensors ${}^{1}T_{i}^{a}$, ${}^{2}T_{ij}^{ab}$ etc.

A common way of solving Eqn. \ref{eq:cc_residuals} is to split residual 
expressions into left and right hand sides to extract amplitudes. One such 
splitting is
\begin{subequations}
\begin{align}
 {}^{1}T_{i}^{a} &= {}^{1}D_{i}^{a} ~ {}^{1}G_{i}^{a}({}^{1}T, {}^{2}T), \\
{}^{2}T_{ij}^{ab} &= {}^{2}D_{ij}^{ab} ~ {}^{2}G_{ij}^{ab}({}^{1}T, {}^{2}T), 
\label{eq:cc_amplitude_equations_b}
\end{align}
\label{eq:cc_amplitude_equations}
\end{subequations}
Here, ${}^1D$ and ${}^2D$ are orbital energy
denominator tensors constructed from diagonal elements of the Fock matrix
$F$:

\begin{subequations}
\begin{align} {}^1D_i^a &= \frac{1}{F_a^a - F_i^a}, \\
{}^{2}D_{ij}^{ab} &= \frac{1}{F_{a}^{a} + F_{b}^{b} - F_{i}^{i} -
F_{j}^{j}}.
\end{align}
\label{eq:denom_definition}
\end{subequations} 

Amplitude equations \ref{eq:cc_amplitude_equations} are solved by iterations 
until a fixed point is found. 

Solving amplitude equations is computationally demanding, and determines a 
very steep cost of Coupled Cluster approach. For example, the evaluation of 
the right hand side of Eq. \ref{eq:cc_amplitude_equations_b} requires 
$O(N^6)$ summations and multiplications per iteration, hence RCCSD method has 
$O(N^6)$ cost. The root of this problem is the need to manipulate high order 
tensors representing the Hamiltonian and excitation amplitudes. This problem, 
however, can be circumvented by using novel techniques of tensor 
decompositions coming from multilinear algebra.\cite{kolda2009tensor} 

\section{Tensor decompositions and wiring diagrams}
