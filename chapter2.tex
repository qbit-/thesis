\chapter{Tensor structured Coupled Cluster}
\label{ch:tcc} 


\section{Motivation}
\label{sec:Introduction} 
Having seeen that both strong and weak electronic correlation can be captured 
by methods based on the Projected Hartree-Fock, let us now turn to another 
powerful family of methods to solve many-body problem, namely, Coupled Cluster 
(CC). Since their introduction in 1984,\cite{} Coupled Cluster methods quickly 
became a "golden standard" of quantum chemistry due to their exceptional 
 ability to capture weak electronic correlation, while having polynomial 
computational cost in basis size. Another attractive property of most CC 
methods are size consistency and size extensivity.\cite{} 

{\color{red} Insert 2 paragraphs here where accuracy of Coupled Cluster is 
discussed. Take it from Irek.}

Coupled Cluster methods parameterize the solution in terms of a reference 
wavefunction $| 0 \rangle$ and a set of excitation operators:
\begin{equation}
 | \phi \rangle  = \exp({}^1T + {}^2T + {}^3T + \ldots) | 0 \rangle
\end{equation}
where ${}^1T, {}^2T, \ldots$ are appropriate single, double, and higher order 
excitations from the reference state $| 0 \rangle$ (usually a HF solution). The 
method with n-body excitations is exact for n-electron systems. In 
practice, however, excitation operator is truncated at doubles due to high 
computational cost, leading to Coupled Cluster with Singles and Doubles methods 
(CCSD).\cite{} Higher order amplitudes can be included approximately
with the help of perturbation theory, such as in the highly accurate and 
well known Coupled Cluster with Singles, Doubles and perturbative 
Triples (CCSD(T)) approach.\cite{}

CC methods are also distinguished by the form of excitation operators, 
leading to Restricted (RCCSD), Unrestricted (UCCSD) and General (GCCSD) 
Coupled Cluster, in the same way as Hartree-Fock methods can be classified. 
A further analogy to HF method is that the solution of conventional Restricted 
Coupled Cluster method has proper $S^{2}$ and $S^{z}$ symmetry, but a wrong 
energy in case of strong correlation.

\begin{figure}[!ht]
\centering
 \rule{0.8\textwidth}{2cm}
 \caption{Restricted Coupled Cluster with Singles and Doubles in different 
correlation regimes. System: $N_{2}$, cc-pVDZ basis set. Exact is FCI solution.}
\end{figure}

This failure of RCCSD in strong correlation regime is one (and probably the 
only) weak side of traditional Coupled Cluster. Another drawback is the 
computational cost, which scales as $O(N^6)$ for RCCSD, where $N$ is a 
size of basis. Both of those problems 
